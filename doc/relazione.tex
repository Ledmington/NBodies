\documentclass[12pt,a4paper,oneside,draft]{article}

\usepackage[italian]{babel}
\usepackage[T1]{fontenc}
\usepackage[utf8]{inputenc}
\usepackage[margin=1in]{geometry}
\usepackage{tikz}

\begin{document}
	
	\title{N-bodies simulation\\Assignment programmazione concorrente e distribuita}
	\author{Filippo Barbari}
	\date{}%no date
	\maketitle
	
	\tableofcontents
	\newpage
	
	\section{Analisi}
	\subsection{Descrizione della simulazione}
	Questa simulazione è una variante della nota "simulazione di $N$ corpi". La simulazione è composta da:
	\begin{itemize}
		\item un numero $N$ di corpi ciascuno dotato di massa ma incorporeo (non ha una dimensione)
		\item un dominio bidimensionale finito di forma rettangolare e allineato con gli assi cartesiani
		\item una forza repulsiva tra i corpi (descritta in dettaglio di seguito)
		\item una forza d'attrito applicata sui singoli corpi
	\end{itemize}
	
	\subsection{Analisi dell'algoritmo}
	\subsection{Analisi delle dipendenze tra dati}
	
	\section{Design}
	\subsection{Using two barriers}
	
	\section{Dettagli implementativi}
	
	\section{Valutazione prestazioni}
	\subsection{Tempi di esecuzione}
	I tempi riportati di seguito fanno riferimento ad un'esecuzione parallela che utilizza 8 thread.
	
	\hfill
	\begin{minipage}{.4\textwidth}
		Senza GUI
		
		\begin{tabular}{|l|l|l|}
			\hline
			\multicolumn{1}{|c|}{\textbf{N. corpi}} & \multicolumn{1}{c|}{\textbf{N. step}} & \multicolumn{1}{c|}{\textbf{Tempo}} \\ \hline
			100 & 1000 & 0,11 \\ \hline
			100 & 10000 & 0,602 \\ \hline
			100 & 50000 & 2,939 \\ \hline
			1000 & 1000 & 1,403 \\ \hline
			1000 & 10000 & 14,139 \\ \hline
			1000 & 50000 & 65,73 \\ \hline
			5000 & 1000 & 47,193 \\ \hline
			5000 & 10000 & 474,334 \\ \hline
			5000 & 50000 & 2351,408 \\ \hline
		\end{tabular}
	\end{minipage}
	\hfill
	\begin{minipage}{.4\textwidth}
		Con GUI
		
		\begin{tabular}{|l|l|l|}
			\hline
			\multicolumn{1}{|c|}{\textbf{N. corpi}} & \multicolumn{1}{c|}{\textbf{N. step}} & \multicolumn{1}{c|}{\textbf{Tempo}} \\ \hline
			100 & 1000 & ??? \\ \hline
			100 & 10000 & ??? \\ \hline
			100 & 50000 & ??? \\ \hline
			1000 & 1000 & ??? \\ \hline
			1000 & 10000 & ??? \\ \hline
			1000 & 50000 & ??? \\ \hline
			5000 & 1000 & ??? \\ \hline
			5000 & 10000 & ??? \\ \hline
			5000 & 50000 & ??? \\ \hline
		\end{tabular}
	\end{minipage}
	\hfill

	\subsection{Speedup}
	\begin{tikzpicture}[scale=0.8, domain=0:8]
		\draw[->] (-0.1,0) -- (8.1,0) node[right] {$x$};
		\draw[->] (0,-0.1) -- (0,8.1) node[above] {$y$};
		\draw[dotted] (-0.1,-0.1) grid (8.1,8.1);
		\draw [color=red] (0,0) -- (8,8); % linear speedup
		\draw (0,0) -- (1,0.9);
		\draw (1,0.9) -- (2,1.9);
		\draw (2,1.9) -- (3,2.8);
		\draw (3,2.8) -- (4,3.5);
		\draw (4,3.5) -- (5,4.4);
		\draw (5,4.4) -- (6,5);
		\draw (6,5) -- (7,5.6);
		\draw (7,5.6) -- (8,6);
	\end{tikzpicture}
	
	\subsection{Strong scaling efficiency}
	\subsection{Profiling}
	Numero di thread?
	Utilizzo memoria?
	\subsection{Throughput}
\end{document}