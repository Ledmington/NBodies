\documentclass[12pt,a4paper,oneside]{article}

\usepackage[italian]{babel}
\usepackage[T1]{fontenc}
\usepackage[utf8]{inputenc}
\usepackage[margin=1in]{geometry}
\usepackage{graphicx}

\usepackage[
	linesnumbered,
	ruled,
	titlenotnumbered,
	italiano
]{algorithm2e}

\begin{document}
	
	\title{N-bodies simulation\\Assignment programmazione concorrente e distribuita}
	\author{Filippo Barbari}
	\date{}%no date
	\maketitle
	
	\tableofcontents
	\newpage
	
	\section{Analisi}
	\subsection{Descrizione della simulazione}
	Questa simulazione è una variante della nota "simulazione di $N$ corpi". La simulazione è composta da:
	\begin{itemize}
		\item un numero $N$ di corpi ciascuno dotato di massa ma incorporeo (non ha una dimensione)
		\item un dominio bidimensionale finito di forma rettangolare e allineato con gli assi cartesiani
		\item una forza repulsiva tra i corpi (descritta in dettaglio di seguito)
		\item una forza d'attrito applicata sui singoli corpi
	\end{itemize}
	
	\subsection{Analisi dell'algoritmo}
	\begin{algorithm}
		\SetAlgorithmName{Bellman-Ford}{}{}
		\KwIn{G, w, s}
		\KwResult{FALSE se il grafo G contiene cicli di costo negativo, TRUE altrimenti}
		Initialize-Single-Source(G,s)\;
		\For{i=1 \textbf{to} |G.V| - 1}{
			\ForEach{edge (u,v) $\in$ G.E}{
				Relax(u,v,w)\;
			}
		}
		\ForEach{edge (u,v) $\in$ G.E}{
			\If{v.d > u.d + w(u,v)}{
				\Return FALSE\;
			}
		}
		\Return TRUE\;
		\caption{L'algoritmo di Bellman-Ford}
	\end{algorithm}
	
	\subsection{Analisi delle dipendenze tra dati}
	Questo algoritmo presenta due dipendenze molto importanti:
	\begin{itemize}
		\item il risultato di una determinata iterazione $i > 0$ dipende dal risultato dell'iterazione $i-1$
		\item il risultato dell'aggiornamento dei parametri dei corpi (accelerazione, velocità e posizione) dipende dal risultato del calcolo delle forze
	\end{itemize}
	
	\section{Design}
	\subsection{Using two barriers}
	
	\section{Dettagli implementativi}
	Per evitare \textit{overhead} non necessari e per semplificare il codice, ogni \texttt{Worker}, prima di cominciare il ciclo vero e proprio, estrae dalla lista dei corpi una lista di puntatori ai corpi su cui deve operare.
	
	\section{Valutazione prestazioni}
	\subsection{Tempi di esecuzione}
	I tempi riportati di seguito fanno riferimento ad un'esecuzione parallela che utilizza 8 thread.
	
	\hfill
	\begin{minipage}{.4\textwidth}
		Senza GUI
		
		\begin{tabular}{|l|l|l|}
			\hline
			\multicolumn{1}{|c|}{\textbf{N. corpi}} & \multicolumn{1}{c|}{\textbf{N. step}} & \multicolumn{1}{c|}{\textbf{Tempo}} \\ \hline
			100 & 1000 & 0,11 \\ \hline
			100 & 10000 & 0,602 \\ \hline
			100 & 50000 & 2,939 \\ \hline
			1000 & 1000 & 1,403 \\ \hline
			1000 & 10000 & 14,139 \\ \hline
			1000 & 50000 & 65,73 \\ \hline
			5000 & 1000 & 47,193 \\ \hline
			5000 & 10000 & 474,334 \\ \hline
			5000 & 50000 & 2351,408 \\ \hline
		\end{tabular}
	\end{minipage}
	\hfill
	\begin{minipage}{.4\textwidth}
		Con GUI
		
		\begin{tabular}{|l|l|l|}
			\hline
			\multicolumn{1}{|c|}{\textbf{N. corpi}} & \multicolumn{1}{c|}{\textbf{N. step}} & \multicolumn{1}{c|}{\textbf{Tempo}} \\ \hline
			100 & 1000 & 0,117 \\ \hline
			100 & 10000 & 0,74 \\ \hline
			100 & 50000 & 3,364 \\ \hline
			1000 & 1000 & 3,101 \\ \hline
			1000 & 10000 & 30,25 \\ \hline
			1000 & 50000 & 143,894 \\ \hline
			5000 & 1000 & 68,841 \\ \hline
			5000 & 10000 & 673,219 \\ \hline
			5000 & 50000 & 3328,55 \\ \hline
		\end{tabular}
	\end{minipage}
	\hfill

	\subsection{Speedup}
	Valori speedup (misurati con 1000 corpi e 10000 step).
	
	\hfill
	\begin{minipage}{.45\textwidth}
		\centering
		\includegraphics[width=\linewidth]{speedup-no-gui}
		\label{fig:speedup-no-gui}
	\end{minipage}
	\hfill
	\begin{minipage}{.45\textwidth}
		\centering
		\includegraphics[width=\linewidth]{speedup-gui}
		\label{fig:speedup-gui}
	\end{minipage}
	\hfill
	
	\subsection{Strong scaling efficiency}
	\hfill
	\begin{minipage}{.45\textwidth}
		\centering
		\includegraphics[width=\linewidth]{sse-no-gui}
		\label{fig:sse-no-gui}
	\end{minipage}
	\hfill
	\begin{minipage}{.45\textwidth}
		\centering
		\includegraphics[width=\linewidth]{sse-gui}
		\label{fig:sse-gui}
	\end{minipage}
	\hfill
	
	\subsection{Profiling}
	Numero di thread?
	Utilizzo memoria?
	\subsection{Throughput}
\end{document}